\documentclass[]{usiinfbachelorproject}
\usepackage{subfig}
\usepackage{caption}
\usepackage{subcaption}
\usepackage{float}
\captionsetup{labelfont={bf}}

\newcommand\subsubsection{\@startsection{subsubsection}{3}{\z@}%
                {-3.25ex\@plus -1ex \@minus -.2ex}%
                {1.5ex \@plus .2ex}%
                {\normalfont\normalsize\bfseries}}
\newcommand\paragraph{\@startsection{paragraph}{4}{\z@}%
                {3.25ex \@plus1ex \@minus.2ex}%
                {-1em}%
                {\normalfont\normalsize\bfseries}}
\author{Your Name}

\title{The Title}
\subtitle{The (optional) subtitle}
\versiondate{\today}
\begin{committee}
%With more than 1 advisor an error is raised...: only 1 advisor is allowed!
\advisor[Universit\`a della Svizzera Italiana, Switzerland]{Prof.}{AdvisorName}{AdvisorSurname}
%You can comment out  these lines if you don't have any assistant
\assistant[Universit\`a della Svizzera Italiana, Switzerland]{Title}{AssistantName1}{AssistanSurname1}
\assistant[Universit\`a della Svizzera Italiana, Switzerland]{Title}{AssistantName2}{AssistanSurname2}
\end{committee}


\abstract {
An abstract describing what the prospectus is all about. ``Omit needless words'' \cite{Stru1899a}: It should fit within the page without making the footer of the title page break the page --- otherwise what kind of abstract would it be?
}


\begin{document}
\maketitle

%%%%%%%%%%%%%%%%%%%%%%%%%
\section{Introduction}
\subsection{Motivation}

Big displays represent a widely used form of mass communication in almost all public places such as squares, subways and schools. They are used in a passive way by pushing un-targeted information, mostly advertisements, to the audience; enslaved by an bidirectional visible communication without any way to personalise the screen's content.

This lack of aimed content bring to a uniqueness of the display's message, forced to be equal for everyone.
In the following pictures you can easily observe the abuse of this strategy by looking at two of the world's biggest squares that are invaded by static, brightly, unavoidable, content from the huge displays.
\begin{figure}[H]
  \centering
  \subfloat[]{\includegraphics[width=0.43\textwidth]{./images/new_york_displays.jpg}}
  \hfill
  \subfloat[]{\includegraphics[width=0.5\textwidth]{./images/piccadilly_displays.jpg}}
\end{figure} 

Such devices make a big impact on a town's shape, they change it's look, colours and feel without adding anything useful. If we look carefully in the pictures we can notice that the only content showed is advertisement. This choice is not related directly to the technology; in the following picture, showing \emph{New York} in the early 60s, we can notice the exactly same advertisement, even from the same companies, that is showed today. Therefore, what is the utility of using displays instead of posters, you may ask.

\begin{figure}[H]
  \centering
  \subfloat[]{\includegraphics[width=0.5\textwidth]{./images/new_york_posters.jpg}}
\end{figure} 
Since the message from the display is shared, or forced, to everyone, it seems logic thinking that everyone should share the screen too, but that is not the case.

A unified message cause a certainly reduction of the screen's utility and purpose, making it, as we stated before, a full passive element. An other example of a badly use of displays may be the ones in the Milan \emph{Stazione Centrale}, showed in the following picture.
% add picture
They are used only to show publicity from the TV's channels loosing all the power that such devices embraces. Would be nice to just walk close to one of these and see our train schedules? Obviously yes. 


\subsection{Goal}

Our goal is to create a bi-lateral channel with the screen: the user must be able to quickly filter the information that he needs in just few seconds without being shelled by tons of useless brands and images. As it is deeply described in the next sections, in our architecture, the user is the active trigger that push and pull it's personal content into and from the screen. This informations can be showed by just walking to the display thanks to proximity sensors such as bluetooth beacons allowing a non blocking information flow where only the relevant content is target to the interested user at the right time.
\\
\\
The information should be pertinent and \emph{fast}, it must be easy for the user to enter the netowkr and become a part of it without loosing any time. Tons of studied shows the connection between poorly user interaction and high learning curve. Since we are addressing a endless mass of people, our application must be easy to start with and addressable, for those we used smart phones as mainly platform
% ADD SOMETHING talk abou speed and user interaface with the smartphone
By doing that we completly transform a tool, such as a display, into a new one, a better one.
% say something of how software can change hardware an it is the main layer
\section{Architecture}
In this section we describe in detail the whole project design, bottom-up. Starting by a briefly description of how we structure all the parts.
\subsection{Introduction}
% INTRODUCTION
Design a working system is never easy. It has to logically mimic the answer to the problem we are trying to solve.

In our case, we are looking for a scalable system where the user, identifier as an active entry, pull and push his information to the display. It must not directly depends on the number of application, or \emph{services}, our network exposes or on the number of display. Specifically the system has to work with $n$ display and $m$ application without any changing in the core. 
% GENERAL OVERVIEW
\\
\\
For such reasons we adopted a \emph{Micro Services} architecture where each element can be removed without effecting the integrity of the network. Services communicate using either synchronous protocols such as HTTP/REST; thus they can be developed and deployed independently. Each service has its own database in order to be decoupled from other services. Such architectures scale faster than a classic monolithic approach, for instance, a new application can be deployed from everywhere really fast by just using the same interface of the existing one. Moreover, each application provider, can use his favourite technologies and hosting platform. In our project we deployed all the services on the same server, but in a real world, as we said,  they are usually physically separated. Since all the applications provide a common interface for pull/push personalization content, we are going to call this set of services the \emph{Application Layer}.\\
\\
All the active logic of the network is handled by another Micro Service, called \emph{Tacita}, that links display, application and users. It's responsibility is to be the glue between the \emph{Application Layer} and all the other elements by storing users' informations, screen's state and available applications.
% TALK about the system that must be anynimous 

\subsection{Architecture Elements}
\subsubsection{General Overview}

We talked superfically of the elements of our Architecture: it is time to deeply describe them one per one. We start from the most important one: The User  
\subsubsection{User}

The User has a fully active role, he is the trigger of the whole system; without him, the network has not reason to exits. He uses his smartphone in order to access the front-end application exposed by the \emph{Appliction Layer} through the \emph{Tacita} platform.

By using it the User can selects his custom settings in order to quickly identify his information on the screens. For instance, a favourite colour can be used for such purpose gaining, at the same time, anonymity, and a quick way to identify his personalized content by just watching the screen. Thus only the targeted client knows which are his information. The app will be deeply analised into the next sections, but, from a user point of view, it is the door to access the array of services.

We talked about the User as a \emph{trigger}, it means that, with his physical being, it \emph{triggers} actions; actions that are universally recognizable by our topology. In our system we use bluetooth beacon near the screens to map them in the space, and, thanks to the smartphone application, we can detect the user walking into the trigger range and show the personalized content into the screen at the right time. 

Timing is a fundamental variable in our ecosystem, if the actions is not handled at the correct time the topology is unreliable and the display is not used correctly, or, not used at all. 
\subsubsection{Display}

Display represent our \emph{tabula rasa} in which a wide array of applications can be showed. By default it can be identify as a \emph{passive} entry, but, combined to the User becomes the second \emph{active} element of our architecture. The two way communication channel between them is created by using a web socket and bluetooth beacon that is notifies the \emph{Tacita} application when a user walk next to a screen. Such information is used to push into a web socket, in which the display is connected, the user unique identifier. As soon as it is received, a request to the running display's application is made in order to get the preferences and showed them; since we are using a common \emph{interface} in the Application layer, each of them expose the same structure to perform CRUD operations, the screen can create the same request structure without being bound to an application .
In our design we decide to only send the minimum amount of data that the display needs, delegating to him the work to fetch further informations making the socket channel faster. The less the display needs, the better. Assuming a user walk near display one, this the notification is pushed into the socked allowing real time notification:

% ADD JSON
Since in each client used a Flux architecture, see SECTION NUMBER, we just need to push a specific action and it will be globally recognized. As soon as it is received, the display, makes a request to the running application asking for the user's preference:
% ADD request example
After the request was successful, obviously, the screen display them.
% add photo of the user display interaction
 An extra communication channel exposed is the touch interface. It allows user to quickly get content without being chained to the \emph{Tacita} application, for instance, we can allowing the user to click somewhere in order to reveal a specific content, for example, a specific bus number.
% talk about design with tach and say that you can find more information at ...
  
\subsubsection{Tacita}
In the previous sections we have said that \emph{Tacita} is the glue of the architecture: it allows communication between users, display and application layer. In order to do so, we create a database's entry for each of these elements and we linked them with the correct relations. Since we are using MySQL we can take advantages of the relation structure that is imposed. A \one to one relation is create between the Display and the Application table, since, a single screen can run only one application at the moment. Also, when the the service is changed, the display automatically notify all the client connected to the web socket of the new state allowing each user to know in real time which application is running where.
% add a photo of the relation in Tacita database

A many to many pivot is used to keep track of witch application is enabled by a user. \emph{Tacita} allows to enable and disable custom services. If an application is turned off  and if a user is near to a display with that application running, the no interaction between the two will happen. Since the client knows the state of a specific screen, it can filter and send only information to the correct entity.
% CHECK

The physical device that make the in-walk communication possible is the bluetooth beacon; thanks to the custom android SDK provided by the manufactor, we can know exactly in which monitoring "region" somebody entered. On the server side, a one to one relation between Beacon and Display is created in order to quickly know, based on the beacon unique Id, which display is linked to. These devices must be putted really close to the machine we want identify in order to increase the location accuracy. Moreover, thanks to our API design, they can be changed at any time by just send the correct request to the server.
% put somewhere else and spiega di piu
We also store the user information, such as email and favourite colour in another table. The email is fetched from the mobile app and used as identifier; by doing so no login is required 

\subsubsection{Transport Micro Service}
The Transport application allows users to navigate the nearby bus station and create preferences through the mobile interface. We decided to gather the data from the Opendata API; a set of well designed endpoint for fetch all kind of transportation, in our case, we used only the buses. However, due to the limit number of request we can made, fixed to three per second and the necessity to have some custom endpoint, we cloned them.

We also have integrated Google Maps API into the screen's front-end; so, after fetching the exactly display's position thanks to web browser geo localization, we can show the estimate time to get to a specific station by walk. Imagine the benefit for a user to have know the minutes he needs to get to a nearby station, also, we can create a visible feedback before it is too late to reach them.

Since we are using API and built-in browser features, our system is scalable. As soon as a display is moved, its position is updated as well with all the other functionalities related to it.
% say how ofter we clone them
% link to opendata
% put some pictures of the scree gui
For what the mobile application, as we talked before, it is made following the required interface. When it is loaded through \emph{Tacita}, the user can create, remove and update his preference. Each of these must have a station as mainly identifier, and an illimited number of buses.
% show some pictures of the preference flow
\subsubsection{Classes Micro Service}
The second implemented application is \emph{Upcoming Classes}, as the name says, it is used to provide the faculty student with useful information such as the course schedule for the next days. Similarly to the Transport application, we had issues with the API, in our case, provided by the University itself.
A single API call to know all the schedules takes more than $3000ms$ as showed:
% metti roba che le loro api fanno schifo
It gets worse if you try to get all the courses for a faculty:
% stessa di sopra

Even if, from the client, we always cache the request, we cannot avoid totwait for the first time. You can already guess why they are so slow by looking at the previous pictures: the response size. They response are heavy due to a poorly model population; for each course that is sent back, tons of unused field are provided. Also, by inspecting a response for a class schedules, we can notice the same huge course object appears, unnecessary, for each schedule object.
Therefore, again, we needed to clone all the API in order to just sent the right amount of information, by doing that, the previous request that tooks more than $3000$ms, now it just needs $METTI VALORE$.

The display's front end application is divided into two main part easily identificable, the calendar and the query engine next to it. The calendar is create using \emph{fullcalendar} jQuery library that does all the dirty work of render and setting up all the events into the corrects slots. 
The courses can be selected thanks to the query engine on the right part of the screen. As soon as the user clieck on the button representing the faculty, it is guided in order to create a valid query using a step by step approach; even if it may be not the faster way, it is the safest since no wrong request can be generated. The procedure is shows in the following storyboard:
% aggiungi storyboard classi

As we did for the Transport Application, we also created a smart phone interface in order to create, remove and edit preferences. Since the interface is always the same we decided to also decide to keep the same design for consistency reason.
% show transport app similar to this one
\subsection{Architecture Interactions}
In the previous part we define in detail each element of our architecture without giving a global overview of how each parts collaborate with the all system. In this section we are going to analize all the interaction, especiatally between user and display, in detail showing each message that the entities exchange, explaing our design decision and analyzing the interactions.
\subsection{Architecture Technologies}
\subsection{Architecture Scalability}
\section{Design}
\subsection{Back-end}
\subsection{Front-end}
\section{Conclusion}
\section{Acknowledges}

\newpage



%%%%%
\bibliographystyle{abbrv}
\bibliography{references}
\end{document}